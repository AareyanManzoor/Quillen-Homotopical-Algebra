\usepackage{fontspec} 
\setmainfont{Open Dyslexic} 
\usepackage{filecontents}

%create a lua script file
\begin{filecontents*}{luaFunctions.lua}

function createReplaceTable()    
    replaceTable = {}

    -- create a table with all ASCII chars
    -- the name and(!) the value of each table item is the ASCII char
    -- this is important if the char shouldn't be replaced
    -- the table have 128 items each filled with the corresponding char
    for i = 1, 128, 1 do    
       replaceTable[string.char(i-1)] = string.char(i-1)
    end
end

function parseString(input)
    outputString = ""

    -- for each char in the given string we replace
    -- the char with the content of the table item
    -- because the table items have the same name like the chars
    -- we have access to the table item via the given char
    for i = 1, string.len(input) do
        char = input:sub(i, i)
        outputString = outputString..rotateString(char)
    end

    tex.print(outputString)
end

function parseFile(fileName)
    -- open file
    local input = io.open('lorem.txt', 'r')

    -- parse each line
    for line in input:lines() do
        parseString(line)
    end
end

function rotateString(c)
    if c == " " then
      return c
    end
    local ang = math.random(-10,10)
    local dp = math.random(-.5,.5)
    return "\\raisebox{" .. dp .. "pt}{\\rotatebox[origin=c]{" .. ang .. "}{" .. c .. "}}"
end

function fillReplaceTable()
    -- here we fill/override the replacements for each ASCII char
    replaceTable["L"] = "\\textbf{\\large L}\\marginpar{\\tiny 'L'(\\stepcounter{counterForL}\\#\\thecounterForL)}"
    replaceTable["o"] = "\\underline{o}"
    replaceTable["e"] = ""
end

\end{filecontents*}    

% read the external lua file to declare the functions,
% but without execute the Lua commands and functions
\directlua{dofile("luaFunctions.lua")}

%create and fill the tables
\directlua{createReplaceTable()}
\directlua{fillReplaceTable()}

% latex commands to execute the lua functions
\def\parseString#1{\directlua{parseString("#1")}}
\def\parseFile#1{\directlua{parseFile("#1")}}
%%%%
\usepackage{sfmath}