\documentclass[../main]{subfiles}

\begin{document}
\begin{theorem} 
  When each of groups $H_{M}^{q} (X,A), H_{\cot}^{q} (X,A)$, and $R^{q_{h_{A}}}(X)$ is defined, it is canonically isomorphic with the Grothendieck sheaf cohomology group $H_{GT}^{q} (X,A)$.
 \end{theorem}
  \begin{proof} 
    We begin by showing that $H_{M}^{q} (X,A) = R^{q_{h_{A}}}(X)$. Let $F$ be the abelian category $(A / X)_{ab}$. $F$ has enough projectives, namely those of the form $P_{ab}$ where $P$ is a projective object $\mathbf A / X$. Hence $s \mathbf{F} $ and $\Ch(\mathbf{F})$ are model categories (see Remark 5 at the end of \S\ref{sec:2.4}) and $N:s \mathbf{F} \longrightarrow  \Ch(\mathbf{F})$ is an equivalence of model categories. The loop and suspension functors on $\Ho (\Ch \mathbf{F} )$ are given very simply by functors $\Omega$ and $\Sigma$ on $\Ch \mathbf{F} $ defined by the formulas
\begin{align*} 
  &(\Sigma X)_{q} = 
  \begin{cases} 
    X_{q-1} & q > 0 \qquad d \Sigma X = - \Sigma dX \\
    0 & q = 0  
  \end{cases} \\[0.5em]
  &(\Omega X)_{q} = 
  \begin{cases} 
    X_{q+1} & q > 0 \qquad d \Omega X = - \Omega dX \\
    \Ker \{d: X_1 \longrightarrow  X_0\} & q = 0 
  \end{cases}
\end{align*}
Let $A[q]$ be the chain complex in $\mathbf{F}$ which is $A$ in dimension $q$ and 0 elsewhere $(A[q] \text{ if } q < 0)$. As \[NK(A,0) = A[0],\quad N\Omega^{q+N}\Sigma^NK(A,0) = A[q],\]
hence 
\begin{align*} 
    H_{M}^{q} (X,A) &= [\mathbf{L}\ab (X), \Omega^{q+N}\Sigma^NK(A,0)] = \pi_{0}((P_\bullet)_{ab},\Omega^{q+N}\Sigma^NK(A,0)) \\[0.5em]
   &= \pi (N(P_\bullet)_{ab}, A[q])= H^{q}\Hom_{\mathbf{F}}(N(P_\bullet)_{ab},A)\\[0.5em]
   &= H^q(\Hom_{\mathbf{F}}((P_\bullet)_{ab}, A)) = H^{q}h_{A}(P_\bullet) =R^{q}h_{A}(X).
\end{align*}

To finish the theorem we need some results about Grothendieck sheaves (\cite{artin_grothendieck_1962}, \cite{verdier_categories_nodate}). Let $\mathbf T$ denote a Grothendieck topology whose underlying category is closed under finite projective limits and has sufficiently many projectives, and where a covering of an object $Y$ in $\mathbf T$ is a family $\mathbf{U} = (Z \longrightarrow  Y)$ consisting of a single effective epimorphism. Eventually we will let $\mathbf T$ be $\mathbf A / X.$ A \defemphi{presheaf} of sets (resp. abelian groups) is a functor $\mathbf{T}^0\longrightarrow \mathbf{Set}$(resp $\mathbf{Ab}$) and a presheaf of sets (resp. abelian groups) is a presheaf $F$ such that for any effective epimorphism $Z \longrightarrow  Y$ the diagram
\[
  F(Y) \longrightarrow  F(Z) \rightrightarrows F(ZX_{Y}Z)
\] 
is exact. 
  
Letting \defemph{Pr} and \defemph{Sh} (resp. \defemph{Prab} and \defemph{Shab}) denote the categories of presheaves and sheaves of sets (resp. abelian groups) we have the diagram
% https://q.uiver.app/?q=WzAsNCxbMiwwLCJcXGRlZmVtcGh7UHJ9Il0sWzAsMCwiXFxkZWZlbXBoe1NofSJdLFswLDIsIlxcZGVmZW1waHtTaGFifSJdLFsyLDIsIlxcZGVmZW1waHtQcmFifSJdLFsxLDAsImkiLDIseyJvZmZzZXQiOjF9XSxbMCwxLCJhIiwyLHsib2Zmc2V0IjoxfV0sWzIsMSwiaiIsMix7Im9mZnNldCI6MX1dLFsxLDIsIlxcdW5kZXJsaW5le1p9IiwyLHsib2Zmc2V0IjoxfV0sWzIsMywiaSIsMix7Im9mZnNldCI6MX1dLFszLDIsImEiLDIseyJvZmZzZXQiOjF9XSxbMCwzLCJcXG1hdGhiYntafSIsMix7Im9mZnNldCI6MX1dLFszLDAsImoiLDIseyJvZmZzZXQiOjF9XV0=
\[\tag{4}\label{eq:2.5.4}
    \begin{tikzcd}
    	{\defemph{Sh}} && {\defemph{Pr}} \\
    	\\
    	{\defemph{Shab}} && {\defemph{Prab}}
    	\arrow["i"', shift right=1, from=1-1, to=1-3]
    	\arrow["a"', shift right=1, from=1-3, to=1-1]
    	\arrow["j"', shift right=1, from=3-1, to=1-1]
    	\arrow["{\mathbf{Z}}"', shift right=1, from=1-1, to=3-1]
    	\arrow["i"', shift right=1, from=3-1, to=3-3]
    	\arrow["a"', shift right=1, from=3-3, to=3-1]
    	\arrow["{\mathbb{Z}}"', shift right=1, from=1-3, to=3-3]
    	\arrow["j"', shift right=1, from=3-3, to=1-3]
    \end{tikzcd}
\]
Here $i$ and $j$ are inclusion functors which are right adjoint functors and the other functors are left adjoint functors. The square of left (resp. right) adjoint functors commutes up to canonical isomorphisms.

We recall the construction of $a$, the associated sheaf functor. If $F \in \Ob\defemph{Pr}$ (resp. $\Ob\defemph{Prab}$), then the $0-$th (resp. $q-$th) Čech cohmology presheaf of $F$ is defined by
\begin{align*}
    \cuhomology^\circ (F)(Y) &= \lim_{\overrightarrow{\mathbf{U}}} \chomology^\circ (\mathbf{U}, F) \\
    \text{(resp. } \cuhomology^q (F)(Y) &= \lim_{\overrightarrow{\mathbf{U}}} \chomology^q (\mathbf{U}, F) \text{)}
\end{align*}
where the limit is taken over the category coverings $\mathbf{U} = (U \longrightarrow Y)$ of $Y$ and where
\[
    \chomology^\circ ((U \longrightarrow Y), F) = \Ker\lbrace F(U) \rightrightarrows F(U\times_YU) \rbrace
\]
(resp. $\chomology^q((U \longrightarrow Y), F) = $ the $q-$th cohomology of the cosimplicial abelian group
\[
    F(U) \rightrightarrows F(U\times_YU) \rightrightrightarrows F(U\times_YU\times_YU) ~\ldots~)
\]
Then $aF = \cuhomology^\circ \cuhomology^\circ (F)$. Given $Y$, choose an effective epimorphism $P \longrightarrow Y$ with $P$ projective; it follows that $(P \longrightarrow Y)$ is cofinal in the category of coverings of $Y$ and hence
\[
    \chomology^\circ(F)(Y) = \Ker\lbrace F(P) \rightrightarrows F(P\times_YP)\rbrace
\]
In particular $\cuhomology^\circ(F)(P) = F(P)$ if $P$ is projective, and hence
\[\tag{5}\label{eq:2.5.5}
    (aF)(P) = F(P)
\]
If $Y$ is arbitrary choose effective epimorphism $P_0 \longrightarrow Y$, $P_l \longrightarrow P_0\times_YP_0$, whence
\begin{align*}
    a(F)(Y) &= \Ker\lbrace(aF)(P_0) \rightrightarrows (aF)(P_l)\rbrace \\
    &= \Ker\lbrace F(P_0) \rightrightarrows F(P_l)\rbrace
\end{align*}
It follows that for $F \in \Ob\defemph{Prab}$, $aF = 0$ if and only if $F(P) = 0$ for all projective $P$. Now if $F' \varrightarrow{u} F \varrightarrow{v} F''$ are maps in \defemph{Shab} with $vu = 0$, then this sequence is exact iff $aH = 0$ where $H = \Ker v/\Image u$ in the category \defemph{Prab}. Hence we have proved


\begin{lemma}\label{lem:2.5.01}
    A sequence $F' \longrightarrow F \longrightarrow F''$ of abelian sheaves is exact iff $F'(P) \longrightarrow F(P) \longrightarrow F''(P)$ is exact for all projective objects $P$.
\end{lemma}


Let $\mathbb{Z}(S)$ denote the free abelian group generated by a set $S$. Then the abelianization functor $\mathbb{Z}$ for presheaves is given by $(\mathbb{Z}F)(Y) = \mathbb{Z}(F(Y))$ for all $Y$ hence combining (\ref{eq:2.5.5}) and the commutativity of (\ref{eq:2.5.5}) we obtain


\begin{lemma}\label{lem:2.5.02}
    If $F$ is a sheaf of sets, then its abelianization $\mathbf{Z}F$ is such that
    \[
        \mathbf{Z}F(P) = \mathbb{Z}(F(P))
    \]
    for all projectives $P$.
\end{lemma}


Let $\uhomology^q : \defemph{Shab} \longrightarrow \defemph{Prab}$ be the $q-$th cohomology presheaf functors. Then $\uhomology^q \quad q \geq 0$ are the right derived functors of $i : \mathbf{Simp} \longrightarrow \mathbf{P}$ and \\$\uhomology^\bullet (F)(Y) = \homology^\bullet (Y, F)$ is the cohomology of $F$ over $Y$. We define a weak equivalence in $s\mathbf{T}$ to be a map $Z_\bullet  \longrightarrow Y_\bullet $ such that for any projective object $P$,\\ $\UHom(P, Y_\bullet ) \longrightarrow \UHom(P, Z_\bullet )$ is a weak equivalence in $\mathbf{Simp}$. This agrees with the definition in \S\ref{sec:2.4}.


\begin{proposition}\label{prop:2.5.01}
    The following are equivalent for a sheaf of abelian groups:
    
    \begin{enumerate}[label=(\roman*)]
        \item $\uhomology^q(F) = 0 \quad q > 0$.

        \item $\chomology^q((U \longrightarrow Y), F) = 0 \quad q > 0$ for all effective epimorphisms $U \longrightarrow Y$.

        \item For any weak equivalence $Z_\bullet  \longrightarrow Y_\bullet $ in $s\mathbf{T}$
            \[
                \homology^\ast (F(Y_\bullet )) \varrightarrow{\sim} \homology^\ast (F(Z_\bullet ))
            \]
    \end{enumerate}
\end{proposition}


A sheaf satisfying the equivalent conditions of Prop. 1 will be called \defemphi{flask}. By (i) any injective sheaf is flask.


\begin{proof}
    (i) $\Longrightarrow$ (iii). Let $h_Y : \mathbf{T}^0 \longrightarrow \mathbf{Set}$ be the functor repesented by $Y$; then $h_Y$ is a sheaf. Let $\mathbf{Z}_Y = \mathbf{Z}h_Y$ so that
    \[
        \Hom_{\defemph{Shab}}(\mathbf{Z}_Y, F) = F(Y)
    \]
    Let $I^\bullet $ be an injective resolution of $F$ in \defemph{Shab} so that $\homology^q(I^\bullet (Y)) = \homology^q(Y, F) = 0$ for all $Y$. Then
    \begin{align*}
        \homology_h^p\homology_v^q\Hom(\mathbf{Z}_Y, I^\bullet ) &= \homology^p\homology^q(Y_\bullet , F) = \begin{cases} \homology^p(F(Y_\bullet )) &q = 0 \\ 0 &q > 0 \end{cases} \\
        \homology_v^p\homology_h^q\Hom(\mathbf{Z}_{Y_\bullet }, I^\bullet ) &= \homology^p\Hom(\homology_q(\mathbf{Z}_{Y_\bullet }), I^\bullet ) = \Ext^p(\homology_q(\mathbf{Z}_{Y_\bullet }), F)
    \end{align*}
    and so we obtain a spectral sequence
    \[
        E_2^{pq} = \Ext^p(\homology_q(\mathbf{Z}_{Y_\bullet }), F) \Longrightarrow \homology^{p + q}(F(Y_\bullet ))
    \]
    and a similar spectral sequence for $Z_\bullet $. Hence we are reduced to showing that $\homology_*(\mathbf{Z}_{Z_\bullet }) \varrightarrow{\sim} \homology_*(\mathbf{Z}_{Y_\bullet })$. By Lemmas \ref{lem:2.5.01} and \ref{lem:2.5.02} we are reduced to showing that $\mathbb{Z}\UHom(P, Z_\bullet ) \longrightarrow \mathbb{Z}\UHom(P, Y_\bullet)$ is a weak equivalence of simplicial abelian groups for each projective $ P_\bullet $  But this is clear since $\Hom(P, Z_\bullet ) \longrightarrow \UHom(P, Y_\bullet )$ is a weak equivalence and since $\pi_\ast (\mathbb{Z}K_\bullet )$, the homology of a simplicial set $K_\bullet $, is a weak homotopy invariant.

    (ii) $\Longrightarrow$ (i). There is a Carten-Leray spectral sequence \\$E_2^{pq} = \cuhomology^p(\uhomology^qF) \Longrightarrow \uhomology^{p + q}F$ \cite[3.5, Ch.I.]{artin_grothendieck_1962}. By assumption $E_2^{p0} = \cuhomology^pF = 0$ for $p > 0$ hence by induction on $n$ one sees that $\uhomology^nF = 0$.

    (iii) $\Longrightarrow$ (ii). $\chomology^q((U \longrightarrow Y), F) = \homology^q(F(Z_\bullet ))$ where $Z_\bullet $ is the object of $s\mathbf{T}$ with \[Z_q = \underbrace{U\times_Y\ldots \times_YU}_{q+1\text{ times}}.\] Regarding $Y$ as a constant simplicial object, $Z_\bullet  \longrightarrow Y$ is a weak equivalence. In effect if $P$ is projective $\Hom(P, U) \longrightarrow \Hom(P, Y)$ is surjective; denoting this by $S \longrightarrow T$ we have that $\UHom(P, Z_\bullet ) \longrightarrow \UHom(P, Y_\bullet )$ is the map
    % https://q.uiver.app/?q=WzAsOCxbMywwLCJTIl0sWzEsMCwiUyBcXHVuZGVyc2V0e1R9e1xcdGltZXN9IFMgXFx1bmRlcnNldHtUfXtcXHRpbWVzfSBTIl0sWzIsMCwiUyBcXHVuZGVyc2V0e1R9e1xcdGltZXN9IFMiXSxbMCwwLCJcXGxkb3RzIl0sWzIsMSwiVCJdLFsxLDEsIlQiXSxbMywxLCJUIl0sWzAsMSwiXFxsZG90cyJdLFsxLDIsIiIsMCx7Im9mZnNldCI6LTJ9XSxbMSwyLCIiLDIseyJvZmZzZXQiOjJ9XSxbMiwwLCIiLDEseyJvZmZzZXQiOjF9XSxbMSwyXSxbMiwwLCIiLDEseyJvZmZzZXQiOi0xfV0sWzEsNV0sWzIsNF0sWzAsNl0sWzUsNF0sWzUsNCwiIiwxLHsib2Zmc2V0IjoyfV0sWzUsNCwiIiwxLHsib2Zmc2V0IjotMn1dLFs0LDYsIiIsMSx7Im9mZnNldCI6MX1dLFs0LDYsIiIsMSx7Im9mZnNldCI6LTF9XV0=
    \[
        \begin{tikzcd}
        	\ldots & {S \underset{T}{\times} S \underset{T}{\times} S} & {S \underset{T}{\times} S} & S \\
        	\ldots & T & T & T
        	\arrow[shift left=2, from=1-2, to=1-3]
        	\arrow[shift right=2, from=1-2, to=1-3]
        	\arrow[shift right=1, from=1-3, to=1-4]
        	\arrow[from=1-2, to=1-3]
        	\arrow[shift left=1, from=1-3, to=1-4]
        	\arrow[from=1-2, to=2-2]
        	\arrow[from=1-3, to=2-3]
        	\arrow[from=1-4, to=2-4]
        	\arrow[from=2-2, to=2-3]
        	\arrow[shift right=2, from=2-2, to=2-3]
        	\arrow[shift left=2, from=2-2, to=2-3]
        	\arrow[shift right=1, from=2-3, to=2-4]
        	\arrow[shift left=1, from=2-3, to=2-4]
        \end{tikzcd}
    \]
    which is a homotopy equivalence by the cone construction.
\end{proof}


\begin{lemma}\label{lem:2.5.03}
    With the notations of (\ref{lem:2.5.02}) $C_\bullet(X) \longrightarrow X$ is a weak equivalence.
\end{lemma}


\begin{proof}
    Let $P$ be projective, as $FSP \longrightarrow P$ is an effective epimorphism it follows that $P$ is a retract of $FSF$. It suffices to show therefore that \[\UHom_{\mathbf{A}}(FB, C_\bullet(X)) \longrightarrow \UHom_{\mathbf{A}}(FB, X)\]
    or \[\UHom_{\mathbf{B}}(B, SC_\bullet(X)) \longrightarrow \UHom_{\mathbf{B}}(B, SX)\]
    is a weak equivalence of simplicial sets. However $SC_\bullet(X) \longrightarrow SX$ is a homotopy equivalence by the ``cone construction''.
\end{proof}


We can now finish the proof of the theorem. Let $\mathbf{T} = \mathbf{A}/X$ and let $I^\bullet $ be a flask resolution of the sheaf $h_A$ and let $P_\bullet  \longrightarrow X$ be a weak equivalence where each $P_q$ is projective. For the double complex $I^\bullet (P_\bullet )$ we have
\[
    \homology_h^p\homology_v^q(I^\bullet (P_\bullet )) = \begin{cases}\homology^p(I^\bullet (X)) &q = 0 \\ 0 &q > 0\end{cases} = \begin{cases}\homology_{GT}^p(X, A) \\ 0 &q > 0\end{cases}
\]
by Prop. \ref{prop:2.5.01} and
\[
    \homology_v^p\homology_h^q(I^\bullet (P_\bullet )) = \begin{cases}\homology^p(h_A(P_\bullet )) &q = 0 \\ 0 &q > 0\end{cases} = \begin{cases}R^ph_a(X) &q = 0 \\ 0 &q > 0\end{cases}
\]
by Lemma \ref{lem:2.5.01}. Thus the two spectral sequences of a double complex degenerate giving $\homology_{GT}^p(X, A) \simeq \homology_{cot}^p(X, A)$ by Lemma \ref{lem:2.5.03} and condition (ii) on the functors (\ref{eq:2.5.3}). \end{proof}


\begin{examples}
    Let $\mathbf{A} = \mathbf{Grp}$ and let $G$ be a group. Then any abelian group object in $\mathbf{A}/G$ is of the form $M\times_TG \varrightarrow{pr_2} G$ where $M$ is a $G$ module and $M\times_TG$ denotes the semi-direct product of M and G. Hence $(\mathbf{A}/G)_{ab}$ is the abelian category of $G$ modules. Moreover if $X \longrightarrow G$ is a group over $G$, then \\$\Hom_{\mathbf{A}/G}(X, M\times_TG) = \Der(X, M)$, the derivative of $X$ with values in $M$ regarded as an $X$ module via the map $X \longrightarrow G$. For each group $X$ over $G$, let $C^q(X, M) = \Hom_{\mathbf{Set}}(X^q, M)$ be the group of $q$ cochain of $X$ with values in $M$ and let $\delta : C^q(X, M) \longrightarrow C^{q + 1}(X, M)$ be the usual coboundary operator. Then
    \[
        0 \longrightarrow \Der(\cdot, M) \longrightarrow C'(\cdot, M) \varrightarrow{\delta} C^2(\cdot, M) \varrightarrow{\delta} \ldots
    \]
    is a flask resolution of the sheaf $h_{M\times_TG}$ on $\mathbf{A}/G$. In effect any weak equivalence of simplicial groups is a homotopy equivalence of sets and the functor $C^q(X, M)$ depend only on the underlying set of $X$; hence $C^q(\cdot, M)$ is flask by Prop. \ref{prop:2.5.01}(iii). On the other hand the sequence is exact by Lemma \ref{lem:2.5.01} and the fact that cohomology of a free group vanishes in dimension $\geq 2$. Thus we find that
    \[
        \homology_{GT}^q(G, M) = \homology_{cot}^q(G, M) = \homology_H^q(G, M) = \begin{cases}\homology^{q + 1}(G, M) &q \geq 1 \\ \Der(G, M) &q = 0\end{cases}
    \]
    where $\homology^\bullet (G, M)$ is the ordinary group cohomology.
\end{examples}


\begin{remarks*}
    \begin{enumerate}[label=(\roman*)]
        \item The preceding example generalizes immediately to cover the usual cohomology of Lie algebras and associative algebras over a field (\cite{barr_acyclic_1966}). Moreover one is lead to the following general picture for cohomology of any kind of universal algebras. Letting $\mathbf{A}$ be a category of universal algebras and $X \in \Ob{\mathbf{A}}$, then an $X$-\defemphi{module} is an abelian group object $A$ in $\mathbf{A}/X$, and the cohomology of $X$ with values in $A$ may be defined to be either $\homology_M^\bullet(X, A)$, $\homology_{GT}^\bullet(X, A)$, or $\homology_{cot}^\bullet(X, A)$ where the cotriple is for example the ``underlying set'' and ``free algebra'' functors $\mathbf{A} \leftrightarrows \mathbf{Set}$. A \defemphi{cochain complex} for computing this cohomology is just a flask resolution of the sheaf $h_A$ on $\mathbf{A}/X$.

        \item The isomorphism $\homology_{GT}^\bullet(X, A) = \homology^\bullet (h_A(P_\bullet ))$ is a special case of a general theorem of Verdier that the Grothendieck sheaf cohomology group $\homology^q(X, F)$ may be computed as $\lim_{\overrightarrow{\mathbf{U}}} \homology^\bullet (\mathbf{U}, F)$ where $\mathbf{U}$ runs over the category of hypercoverings of $X$ for the topology. In effect $P_\bullet  \longrightarrow X$ is cofinal in this category of hypercoverings. See \cite{grothendieck_theorie_1972} especially, exposé V, appendice.
    \end{enumerate}
\end{remarks*}
\end{document}