\documentclass[../main]{subfiles}

\begin{document}
\section{\texorpdfstring{$s\mathbf{A}$}{sA} as a model category}\label{sec:2.4}

Let $\mathbf{A}$ be a category closed under finite limits. A map $f \colon X \longrightarrow Y$ is said o be an \defemph{effective epimorphism} if for any object $T$ the diagram of sets 
\[
\begin{tikzcd}
    {\Hom(Y, T)}
    \arrow[r, "f^*"]
    &
    {\Hom(X, T)}
    \arrow[r, "\pr^*_1", shift left]
    \arrow[r, "\pr^*_2"', shift right]
    &
    {\Hom(X \times_Y X, T)}
\end{tikzcd}
\]
is exact. We shall say that an object $P$ of $\mathbf{A}$ is \defemph{projective} if \\$\Hom(P, X) \longrightarrow \Hom(P, Y)$ is surjective whenever $X \longrightarrow Y$ is an effective epimorphism and that $\mathbf{A}$ has \defemph{sufficiently many projectives} if for any object $X$ there is a projective $P$ and an effective epimorphism $P \longrightarrow X$. If $\mathbf{A}$ is closed under inductive limits, we call an object $X$ \defemph{small} if $\Hom(X, -)$ commutes with \defemph{filtered} inductive limits, and call a class $\mathbf{U}$ of objects of $\mathbf{A}$ a class of \defemph{generators} if for every object $X$ there is an effective epimorphism $Q \longrightarrow X$, where $Q$ is a direct sum of copies of members of $\mathbf{U}$. 

\begin{theorem}
\label{thm:2.4}
Let $\mathbf{A}$ be a category closed under finite limits and having sufficiently many projectives. Let $s\mathbf{A}$ be the simplicial category of simplicial objects over $\mathbf{A}$. Define a map $f$ in $s\mathbf{A}$ to be a fibration (resp.\ weak equivalence) if $\UHom(P, f)$ is a fibration (resp.\ weak equivalence) in $\mathbf{Simp}$ for each projective object $P$ of $\mathbf{A}$, and a cofibration if $f$ has the LLP with respect to the class of trivial fibrations. Then $s\mathbf{A}$ is a closed simplicial model category if $\mathbf{A}$ satisfies one of the following extra conditions:
\begin{enumerate}
    \item[($\ast$)] 
    Every object of $s\mathbf{A}$ is fibrant. 
    
    \item[($\ast\ast$)] 
    $\mathbf{A}$ is closed under inductive limits and has a set of small projective generators.
\end{enumerate}
\end{theorem}

Here, and in the following, objects of $\mathbf{A}$ will be identified with constant simplicial objects. For the rest of this section $\mathbf{A}$ will denote a category closed under finite limits and having sufficiently many projectives. We will not use conditions ($\ast$) and ($\ast\ast$) until we absolutely have to. We first make some remarks about the theorem. 

\begin{proposition}
\label{prop:2.4.01}
Suppose that every object $X$ of $\mathbf{A}$ is a quotient of a cogroup object $C$ (i.e.\ there exists an effective epimorphism $C \longrightarrow X$). Then $\mathbf{A}$ satisfies ($\ast$).
\end{proposition}

\begin{proof}
Given $X \in \Ob s\mathbf{A}$ and a projective object $P$ of $\mathbf{A}$, choose an effective epimorphism $C \longrightarrow P$ where $C$ is a cogroup object. Then $P$ is a retract of $C$, so $\UHom(P, X)$ is a retract of $\UHom(C, X)$ which is a group complex. By \ref{Moore}, $\UHom(C, X)$ is a Kan complex hence so is $\UHom(P, X)$, and hence $X$ is fibrant. 
\end{proof}

\begin{remarks*}
\begin{enumerate}
    \item By a theorem of Lawvere \cite{lawvere_functorial_1963} a category closed under inductive limits and having a single small projective generator $U$ is equivalent to the category of universal algebras with a specified set of finitary operations and identities in such a way that $U$ corresponds to the free algebra on one generator. Hence the theorem applies when $\mathbf{A}$ is the category of rings, monoids, etc. One may show that effective epimorphism = set-theoretically surjective map in this case. 
    
    \item The category of profinite groups satisfies ($\ast$) but not  ($\ast\ast$). The free profinite group generated by a profinite set is both projective and a cogroup object in this category and every object is a quotient of such an object. 
\end{enumerate}
\end{remarks*}

The rest of this section contains the proof of Theorem \ref{thm:2.4}.

\begin{proposition}
\label{prop:2.4.02}
Let $\mathbf{A}$ be a category closed under finite limits and having sufficiently many projectives. Then $X \longrightarrow Y$ is effective epimorphism $\iff$ $\Hom(P, X) \longrightarrow \Hom(P, Y)$ is surjective for every projective object $P$. 
\end{proposition}

\begin{proof}
($\implies$) is by definition. For ($\impliedby$) we first establish three properties of effective epimorphisms which hold without assuming $\mathbf{A}$ has enough projectives. It is clear that $f \colon X \longrightarrow Y$ is an effective epimorphism iff for any object $T$ and map $\alpha \colon X \longrightarrow T$ there is a unique $\beta \colon Y \longrightarrow T$ with $\beta f = \alpha$ provided $\alpha$ satisfies the necessary condition that $\alpha u = \alpha v$ whenever $u, v \colon S \rightrightarrows X$ are two maps such that $fu = fv$. 

\begin{enumerate}[label = (\arabic*)]
    \item  If $f \colon X \longrightarrow Y$ has a section $s \colon Y \longrightarrow X$ with $fs = \id_y$, then $f$ is an effective epimorphism. 
\end{enumerate}

In effect, given $\alpha \colon X \longrightarrow T$ satisfying the necessary condition let \\$\beta = \alpha s \colon Y \longrightarrow T$. As $sf, \id_X \colon X \rightrightarrows X$ are two maps with $f(sf) = f(\id_X)$ we have $\beta f = \alpha {sf} = \alpha$. $\beta$ is clearly unique. 

\begin{enumerate}[label = (\arabic*)]
\setcounter{enumi}{1}
    \item  If $X \varrightarrow f Y \varrightarrow g Z$ are maps, where $gf$ is an effective epimorphism and $f$ is an epimorphism, then $g$ is an effective epimorphism.
\end{enumerate}

Given $\alpha \colon Y \longrightarrow T$ with $\alpha u = \alpha v$ whenever $u,v \colon S \longrightarrow Y$ and $gu = gv$, it follows that $\alpha f \colon X \longrightarrow T$ has the property that $\alpha {fu} = \alpha {fv}$, whenever $u,v \colon S \longrightarrow Y$ and $gfu = gfv$. As $gf$ is an effective epimorphism, there is a unique map $\beta \colon Z \longrightarrow T$ with $\beta_{gf} = \alpha {f}$. As $f$ is an epimorphism $\beta g = \alpha$.

\begin{enumerate}[label = (\arabic*)]
\setcounter{enumi}{2}
    \item  If $X \varrightarrow f Y \varrightarrow g Z$ are maps, where $g$ is an effective epimorphism and $f$ has a section $s$, then $gf$ is an effective epimorphism.
\end{enumerate}

In effect given $\alpha \colon X \longrightarrow T$ satisfying the necessary conditions that it factors through $gf$, it in particular satisfies the necessary conditions for factoring through $f$. By (1).\ there is a unique $\beta$ with $\beta f = \alpha$ given by $\beta = \alpha s$. Suppose $u, v \colon S \rightrightarrows Y$ are such that $gu = gv$. Then $gfsu = gfsv$ so $\alpha {su} = \alpha {sv}$ or $\beta_u = \beta_v$. Hence since $g$ is an effective epimorphism there is a unique $\gamma$ with $\gamma g = \beta$ and hence a unique $\gamma$ with $\gamma gf = \alpha$. Thus $gf$ is an effective epimorphism.

Now suppose that $f \colon X \longrightarrow Y$ has $\Hom(P, X) \longrightarrow \Hom(P, Y)$ surjective for all projective objects $P$. Choose an effective epimorphism $u \colon P \longrightarrow X$ with $P$ projective. As $fu$ has the same property as $f$ we are reduced by (2).\ to the case where $X$ is projective. Choose an effective epimorphism $v \colon Q \longrightarrow Y$ with $Q$ projective. As $X$ is projective there is a map $\alpha \colon X \longrightarrow Q$ with $v\alpha = f$ and by the property of $f$ there is a map $\beta \colon Q \longrightarrow X$ with $f \beta = v$. The maps $\alpha$ and $\beta$ yield sections of the maps $\pr_1$ and $\pr_2$ in 
\[
\begin{tikzcd}
    X \times_Y Q
    \arrow[dd, "\pr_1"']
    \arrow[rr, "\pr_2"]
    &&
    Q
    \arrow[dd, "v"]
    \\\\
    X
    \arrow[rr, "f"']
    &&
    Y
\end{tikzcd}
\]

By (1).\ and (3).\ $v\pr_2 = f\pr_1$ is an effective epimorphism and so by (2).\ $f$ is an effective epimorphism. 
\end{proof}

\begin{corollary*}
The class of effective epimorphisms in $\mathbf{A}$ is closed under composition and base change and it contains all isomorphisms. If $gf$ is an effective epimorphism so is $g$. 
\end{corollary*}

In particular, the effective epimorphisms are universally effective.

\begin{proposition}
\label{prop:2.4.03}
Any map $f$ may be factored $f = pi$ where $i$ is a cofibration and where $p$ is a trivial fibration. 
\end{proposition}

\begin{proof}
Given $f \colon X \longrightarrow Y$ construct a diagram
\[
\begin{tikzcd}
    X
    \arrow[rr, "j_0"]
    \arrow[ddrr, "f"']
    &&
    Z^0 
    \arrow[rr, "j_1"]
    \arrow[dd, "p_0"]
    &&
    Z^1
    \arrow[rr, "j_2"]
    \arrow[ddll, "p_1"]
    &&
    \dots
    \\\\&&
    Y
\end{tikzcd}
\]
as follows. Let $Z^{-1} = X$, $p^{-1} = f$ and having obtained $p_{n-1} \colon Z^{n-1} \longrightarrow Y$, choose a projective object $P_n$ of $\mathbf{A}$ and a map $(\alpha, \beta)$ so that 
\begin{equation}\tag{1}
\label{eq:2.4.1}
\begin{tikzcd}[column sep = 1.4in]
    P_n \vee (Z^{n-1})^{\Delta(n)} 
    \arrow[r, "{(\alpha, \beta)+ (p_{n-1}^{\Delta(n)}, (Z^{n-1})^{i_n})}"] 
    &
    Y^{\Delta(n)} \times_{Y^{\overset{\bullet}{\Delta(n)}}} (Z^{n-1})^{\overset{\bullet}{\Delta(n)}}
\end{tikzcd}
\end{equation}
is an effective epimorphism in dimension $0$, where $i_n \colon \overset{\bullet}{\Delta(n)} \longrightarrow \Delta(n)$ is the canonical inclusion. Now define the map $j_n$ by a cocartesian diagram
\begin{equation}\tag{2}
\label{eq:2.4.2}
\begin{tikzcd}
    P_n \otimes \overset{\bullet}{\Delta(n)} 
    \arrow[rr, "P_n \otimes i_n"]
    \arrow[dd, "\beta"']
    &&
    P_n \otimes \Delta(n)
    \arrow[dd, "\In_2"]
    \\\\
    Z^{n-1}
    \arrow[rr, "\In_1 = j_n"']
    &&
    Z^n
\end{tikzcd}
\end{equation}
and let $p_n \colon Z^n \longrightarrow Y$ be the unique map with $p_nj_n = p_{n-1}$ and $p_n \In_2 = \alpha$.

As $i_n \colon \overset{\bullet}{\Delta(n)} \longrightarrow \Delta(n)$ is an isomorphism in dimensions $<n$ so is $j_n$, hence $\varinjlim Z^n = Z$ exists and we may define map $X \varrightarrow i Z \varrightarrow p Y$ by $i = \lim J_n \dots J_0$, $p = \lim p_n$. It is clear that $P_n \otimes i_n$ in (\ref{eq:2.4.2}) is a cofibration, hence each $j_n$ and hence $i$ is a cofibration. To see that $p$ is a trivial fibration it suffices to show that 
\[(P^{\Delta(n)}, Z^{i_n}) \colon Z^{\Delta(n)} \longrightarrow Y^{\Delta(n)} \times_{Y^{\overset{\bullet}{\Delta(n)}}} (Z^{n-1})^{\Delta(n)}\] is an effective epimorphism in dimension $0$. Consider the diagram
\[
\begin{tikzcd}
    P_n \vee (Z^{n-1})^{\Delta(n)}
    \arrow[rr]
    \arrow[dd, "\beta + j_n^{\Delta(n)}"']
    &&
    Y^{\Delta(n)} \times_{Y^{\overset{\bullet}{\Delta(n)}}} (Z^{n-1})^{\overset{\bullet}{\Delta(n)}}
    \arrow[dd, "{(\id, k_{n-1}^{\overset{\bullet}{\Delta(n)}}})"]
    \\ \\
    (Z^n)^{\Delta(n)}
    \arrow[r, "k_n^{\Delta(n)}"']
    &
    Z^{\Delta(n)}
    \arrow[r, "{(p^{\Delta(n)}, Z^{i_n})}"']
    &
    Y^{\Delta(n)} \times_{Y^{\overset{\bullet}{\Delta(n)}}} Z^{\Delta(n)}
\end{tikzcd}
\]
where the top map is the effective epimorphism (\ref{eq:2.4.1}), and where $k_q = \lim j_n \dots j_{q+1}$. $k_{q+1}$ is an isomorphism in dimension $<n$, hence $(\id, k_{n-1}^{\overset{\bullet}{\Delta(n)}})$ is an isomorphism in dimension zero. By the corollary of Prop.\ \ref{prop:2.4.02}, $(p^{\Delta(n)}, Z^{i_n})$ is an effective epimorphism in dimension $0$.
\end{proof}

\begin{proof}[Proof of Theorem \ref{thm:2.4}]
($\ast$) This is exactly the same as the proof in \S\ref{sec:2.3} for $\mathbf{Top}$ and $\mathbf{SimpGrp}$, so we present an outline only. If $f \colon A \longrightarrow B$ is a map, then as $A$ and $B$ are fibrant, $A \varrightarrow i A \times_B B^I \varrightarrow p B$ is a factorization of $f$ into a strong deformation retract map followed by a fibration. The homotopy equivalence $i$ in $s \mathbf{A}$ is carried by $\UHom(P, -)$ into a homotopy equivalence in $\mathbf{Simp}$; hence $i$ is a weak equivalence in $s\mathbf{A}$. If $f$ has the LLP with respect to fibrations, $f$ is a cofibration and a retract of $i$; hence $f$ is a trivial cofibration. Conversely, if $f$ is a trivial cofibration, \ref{1.1.M5} implies $p$ is a trivial fibration so $f$ is a retract of $i$; hence $f$ is a strong deformation retract map, so by \ref{SM7a}, $f$ has the LLP with respect to the fibrations. With this we have \ref{1.1.M6}, hence \ref{SM7}. Finally, \ref{1.1.M2} results from Prop.\ \ref{prop:2.4.03} for the cofibration--trivial fibration case and for the other case one uses this case to write $i = qj$, $j$ cofibration, $q$ trivial fibration, whence $f = (pq)j$ is a factorization where $j$ is a trivial cofibration and $pq$ is a fibration.

($\ast\ast$) Let $\mathbf{U}$ be a set of small projective generators for $\mathbf{A}$. Then the retract argument used in the proof of Prop.\ \ref{prop:2.4.01} shows that a map $f$ in $s\mathbf{A}$ is a fibration or weak equivalence iff $\UHom(P, f)$ is so is $\mathbf{Simp}$ for all $P \in \mathbf{U}$. In particular, the fibrations are characterized by the RLP with respect to the set of maps $P \otimes V(n, k) \longrightarrow P \otimes \Delta(n)$ for each $P \in \mathbf{U}$ and $0 \leq k \leq n >0$. However $P \otimes V(n, k)$ is small in $s \mathbf{A}$  since $P$ is small in $\mathbf{A}$, hence the small object argument implies that any map $f$ may be factored $f = pi$ where $p$ is a fibration and $i$ has the LLP with respect to all fibrations. We must show that $i$ is a weak equivalence. 

For this purpose, we shall use Kan's $\Ex^\infty$ functor \cite{kan_css_1957}. We recall that $(\Ex K)_n$ is the projective limit in the category of sets of a finite diagram involving $K_n$, $K_{n-1}$ and the face operators of $K$. AS $\mathbf{A}$ is closed under finite limits, we may define $\Ex \colon s \mathbf{A} \longrightarrow s \mathbf{A}$ by the formula 
\begin{equation}\tag{3}
\label{eq:2.4.3}
    \UHom(A, \Ex X) = \Ex \UHom(A, X)
\end{equation}
for all $A \in \Ob \mathbf{A}$, $X \in \Ob s \mathbf{A}$. The natural map $K \longrightarrow \Ex K$ in $\mathbf{Simp}$ extends to a map $X \longrightarrow \Ex X$, and hence we may define $\Ex^\infty(X) = \varinjlim \Ex^n(X)$ and a map $\epsilon_X \colon X \longrightarrow \Ex^\infty(X)$. If $P \in \mathbf{U}$, then as $P$ is small
\[\UHom(P, \Ex^\infty X) = \Ex^\infty \UHom(P,X).\] Therefore $\Ex^\infty X$ is fibrant and $\epsilon_X \colon X \longrightarrow \Ex^\infty X$ is a weak equivalence.

Now suppose that $i \colon A \longrightarrow B$ has the LLP with respect to fibrations. Then we may lift successively in 
\[
\begin{tikzcd}
    A
    \arrow[rr, "\epsilon_A"]
    \arrow[dd, "i"']
    &&
    \Ex^\infty A
    \arrow[dd]
    \\\\
    B
    \arrow[rr]
    \arrow[uurr, dashed, "u"]
    &&
    e
\end{tikzcd}
\qquad
\begin{tikzcd}[column sep = huge]
    A
    \arrow[rr, "s\epsilon_B i"]
    \arrow[dd, "i"']
    &&
    (\Ex^\infty B)^{\Delta(1)}
    \arrow[dd, "{(j_0, j_1)}"]
    \\\\
    B
    \arrow[rr, "{(\epsilon_B, (\Ex^\infty i)u)}"']
    \arrow[uurr, dashed, "H"]
    &&
    (\Ex^\infty B) \times (\Ex^\infty B)
\end{tikzcd}
\]
obtaining the formulas $u i = \epsilon_A$, $(\Ex^\infty i) u \sim \epsilon_b$, $\epsilon_B i = (\Ex^\infty i) \epsilon_A$. Let $P \in \mathbf{U}$  and apply the functor $\gamma \circ \UHom(P,-)$ where $\gamma$ is the canonical localization map $\mathbf{Simp} \longrightarrow \Ho \mathbf{Simp}$. It follows that $\gamma \UHom(P,i)$ is an isomorphism hence (Prop. \ref{prop:1.5.1}) $\UHom(P, i)$ is a weak equivalence. Thus $i$ is a weak equivalence and we have proved that a map with the LLP with respect to the fibrations is a trivial cofibration. Conversely if $f$ is a trivial cofibration we may factor $f = pi$ where $p$ is a fibration and $i$ has the LLP with respect to the fibrations; by what we have just shown $i$ is a weak equivalence, hence $p$ is trivial, so $f$ is a retract of $i$ and hence has the LLP with respect to the fibrations. This proves half of \ref{M6} and \ref{1.1.M2}; the other is similar using Prop.\ \ref{prop:2.4.03}. \ref{M6}
implies \ref{SM7} and \ref{1.1.M5} is clear, so the theorem is proved.
\end{proof}

\begin{remarks*}
\begin{enumerate}
    \item Some extra conditions on $\mathbf{A}$ like ($\ast$) or ($\ast\ast$) is necessary since the category of simplicial finite sets fails to satisfy \ref{1.1.M2}. In effect there are simplicial finite sets with infinite homotopy groups. 
    
    \item If the map $\varnothing \longrightarrow X$ in $s\mathbf{A}$ is factored $\varnothing \varrightarrow i Z \varrightarrow p X$ where $p$ is a trivial fibration and $i$ is a cofibration, then it is easily seen by using Prop.\ \ref{prop:2.2.4} that this factorization is unique up to simplicial homotopy over $X$. Now for $Z \longrightarrow X$ to be a trivial fibration is the analogue of $Z$ being a resolution of $X$, while for $Z$ to be cofibrant is the analogue of $Z$ being a complex of projective objects. Hence Prop.\ \ref{prop:2.4.03} asserts for $s\mathbf{A}$ the existence of projective resolutions and so one may define derived functors for $\mathbf{A}$ even when $\mathbf{A}$ does not satisfy ($\ast$) or ($\ast\ast$).
    
    \item It is worthwhile noting that $(X^{\overset{\bullet}{\Delta(n)}})_0 = (\cosk_{n-1} X)_n$ where $\cosk_q$ is the $q$-th coskeleton functor of Verdier \cite{verdier_categories_nodate}. Consequently a trivial fibration $X \longrightarrow A$ where $A$ is an object of $\mathbf{A}$ is the same as hypercovering of $A$ for the Grothendieck topology whose covering families consist of single maps $\{v \longrightarrow u\}$ which are effective epimorphisms. We will discuss this in  the next section.
    
    \item When $\mathbf{A}$ is a category of universal algebras (see Remark 1 after Prop.\ \ref{prop:2.4.01}), then the $P_n$ in the proof of Prop.\ \ref{prop:2.4.03} may be chosen to be free algebras, and so the map $X \varrightarrow i Z$ is \defemph{free} in the following sense: there are subsets $C_q \subset Z_q$ for each $q$ such that 
    \begin{enumerate}[(i)]
        \item $\eta^* C_p \subset C_q$ whenever $\eta \colon [q] \longrightarrow [p]$ is a surjective monotone map,
        
        \item $f_q + g_q \colon X_q \vee FC_q \longrightarrow Z_q$ is an isomorphism for all $q$, where $FC_q$ is the free algebra generated by $C_q$ and $g_q\colon FC_q \longrightarrow Z_q$ is the unique algebra map which is the identity on $C_q$.
    \end{enumerate}
    Conversely, one may show \cite[Thm.\ 6.1]{kan_css_1957-1} that any free map $X \varrightarrow i Z$ may be factored \[X \longrightarrow Z^0 \longrightarrow Z^1 \longrightarrow \dots \longrightarrow Z\] where there are co-cartesian squares (\ref{eq:2.4.2}) with $P_n$ free and hence any free map is a cofibration. Furthermore given a cofibration $f$ we may factor it $f = pi$ where $i$ is free and $p$ is a trivial fibration: then $f$ is a retract of $i$ hence \defemph{a map is a cofibration iff it is a retract of a free map}.
    
    \item If $\mathbf{A}$ is an abelian category with sufficiently many projective objects, then Theorem \ref{thm:2.4} endows $s\mathbf{A}$ with the structure of a closed model category. On the other hand by Dold--Puppe \cite{dold_homologie_1961} the normalization functor\\ $N \colon s\mathbf{A} \longrightarrow \Ch \mathbf{A}$, the category of chain complexes in $\mathbf{A}$ is an equivalence of categories, and moreover the simplicial homotopy relation on maps in $s\mathbf{A}$ corresponds to the chain homotopy relation on maps in $\Ch\mathbf{A}$. The corresponding closed model category structure on $\Ch\mathbf{A}$ may be described as follows: Weak equivalences are maps inducing isomorphisms on homology groups (since $\homology(NX) = \pi X$) and fibrations are maps which are epimorphisms in positive degrees (straightforward generalization of Prop.\ \ref{prop:2.3.1} to abelian categories). Finally cofibrations are monomorphisms whose cockerels are dimension-wise projective. In effect what is called the fundamental theorem of homological algebra amounts essentially to the following: (i) any monomorphism with dimension-wise projective cokernel has the LLP with respect to trivial fibrations and (ii) any map $f$ may be factored $f = pi$ where $p$ is a trivial fibration and $i$ is a monomorphism with dimension-wise projective cokernel.

    As the class of monomorphisms with dimension-wise projective cokernels is closed under retracts, it is seen to be the class of cofibrations by a retract argument.
\end{enumerate}
\end{remarks*}

\end{document}