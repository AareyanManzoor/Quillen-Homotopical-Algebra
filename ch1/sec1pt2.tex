\documentclass[../main]{subfiles}

\begin{document}
If $A$ and $B$ are objects of $\cat C$ we let $\pi^r(A, B)$ (resp. $\pi^l(A, B)$) be the set of equivalence classes of $\Hom(A, B)$ with respect to the equivalence relation generated by $\RHom$ (resp. $\LHom$). When $A$ cofibrant and $B$ is fibrant, in which case $\LHom$ and $\RHom$ coincide and are already equivalence relations by Lemmas~\ref{lem:1.1.4}, \ref{lem:1.1.5}(i) and their duals, we shall denote the relation by $\sim$, call it homotopy and let $\pi_0(A, B)$ or simply $\pi(A, B)$ be the set of equivalence classes. 

\begin{lemma}
\label{lem:1.6}
If $A$ is cofibrant, then composition in $\cat C$ induces a map \newline$\pi^r(A, B) \times \pi^r(B, C) \longrightarrow \pi^r(A, B)$.
\end{lemma}

\begin{proof}
It suffices to show that if $f, g \in \Hom(A, B)$, $u \in \Hom(B, C)$ and $f \RHom g$ then $uf \RHom ug$, which is Lemma~\ref{lem:1.1.5}(iii), and that if $u, v \in \Hom(B, C)$,\\ $f \in \Hom(A, B)$, and $u \RHom v$, then $uf \RHom vf$, which is immediate from the definition.  
\end{proof} 

\begin{lemma}
\label{lem:1.1.7}
Let $A$ be cofibrant and let $p : X \longrightarrow Y$ be a trivial fibration. Then $p$ induces a bijection $p_\ast : \pi^l(A, X) \varrightarrow{\sim} \pi^l(A, Y)$.
\end{lemma} 

\begin{proof}
The map is well--defined since $f \LHom g \implies pf \LHom pg$ is immediate from the definition. The map is surjective by \ref{1.1.M1}. By Lemma~\ref{lem:1.1.4} if $f, g \in \Hom(A, X)$ and $pf, pg$ represent the same element of $\pi^l(A, Y)$, then there is a left homotopy $h : A \times I \longrightarrow Y$ from $pf$ to $pg$. If $H$ is a lifting in 
%TODO: label (10)
\[\tag{10}\label{diag:1.1.10}
    \begin{tikzcd}
    A \vee A \arrow[dd, "\partial_0 + \partial_1"'] \arrow[rr, "f +g"] &  & X \arrow[dd, "p"] \\
                                                                       &  &                   \\
    A \times I \arrow[rr, "h"] \arrow[rruu, "H", dashed]               &  & Y                
    \end{tikzcd}
\]
then $H$ is a left homotopy from $f$ to $g$. This shows that $p_\ast$ is injective.
\end{proof} 

Let ${\cat C}_c$, ${\cat C}_f$, and ${\cat C}_{cf}$ be the full subcategories consisting of the cofibrant, fibrant, and both fibrant and cofibrant objects of $\cat C$ respectively. By Lemma~\ref{lem:1.6} we may define a category $\pi \cat C_c$ with the same objects as ${\cat C}_c$, with \newline$\Hom_{\pi \cat C_c}(A, B) = \pi^r(A, B)$ and with the composition induced from that of $\cat C$. If we denote the right homotopy class of a map $f : A \longrightarrow B$ by $\xoverline f$ we obtain a functor ${\cat C}_c \longrightarrow \pi {\cat C}_c$ given by $X \longrightarrow X$, $f \longrightarrow \xoverline f$. Similarly largely by the dual of Lemma~\ref{lem:1.6} we may define $\pi \cat C_f$ (resp. $\pi \cat C_{cf}$) to be the category with the same objects as ${\cat C}_f$ and with $\pi^l(A, B)$ (resp. $\pi(A, B)$) as maps from $A$ to $B$. 

\begin{definition}\label{def:1.1.5}
Let $\cat C$ be an arbitrary subcategory and let $S$ be a subclass of the class of maps of $\cat C$. By the \defemphi{localization} of $\cat C$ with respect to $S$ we mean a category $S^{-1} \cat C$ together with a functor $\gamma : \cat C \longrightarrow S^{-1} \cat C$ having the following universal property: For every $s \in S$, $\gamma(s)$ is an isomorphism; given any functor $F  : \cat C \longrightarrow \cat B$ with $F(s)$ an isomorphism for all $s \in S$, there is a unique functor $\theta : S^{-1} \cat C \longrightarrow \cat B$ such that $\theta \circ \gamma = F$. 
\end{definition}

Except for set--theoretic difficulties the category $S^{-1} \cat C$ exists and may be constructed by mimicking the construction of the free group (see Gabriel--Zisman \cite{gabriel_calculus_1967}).

\begin{definition}\label{def:1.1.6}
Let $\cat C$ be a model category. Then the \defemph{homotopy category}\index{homotopy!\indexline category} of $\cat C$ is the localization of $\cat C$ with respect to the class of weak equivalences and is denoted by $\gamma : \cat C \longrightarrow \Ho \cat C$. $\quad\gamma_c : \cat C_c \longrightarrow \Ho \cat C_c$ (resp. $\gamma_f : \cat C_f \longrightarrow \Ho \cat C_f$) will denote the localization of $\cat C_c$ with respect to the class of maps in $\cat C_c$ (resp. $\cat C_f$) which are weak equivalences in $\cat C$. We sometimes use the notation $[X ,Y]$ for $\Hom_{\Ho \cat C}(X, Y)$.
\end{definition} 

\begin{lemma}
\label{lem:1.1.8}
\begin{itemize}
    \item[(i)] Let $F : \cat C \longrightarrow \cat B$ carry weak equivalences in $\cat C$ into isomorphisms in $\cat B$. If $f \LHom g$ or $f \RHom g$, then $F(f) = F(g)$ in $\cat B$.
    \item[(ii)] Let $F : \cat C_c \longrightarrow \cat B$ carry weak equivalences in $\cat C_c$ into isomorphisms in $\cat B$. If $f \RHom g$, then $F(f) = F(g)$ in $\cat B$. 
\end{itemize} 
\end{lemma} 

\begin{proof}
\begin{itemize}
    \item[(i)] Let $h : A \times I \longrightarrow B$ be a left homotopy from $f$ to $g$. As $\sigma$ is a weak equivalence, $F(\sigma)$ is an isomorphism. As \[F(\sigma) F(\partial_0) = F(\sigma) F(\partial_1) = \id_A\implies F(\partial_0) = F(\partial_1)\] and so \[F(f) = F(h) F(\partial_0) = F(h) F(\partial_1) = F(g).\]
    \item[(ii)] The proof is the same same as (i) since by Lemma~\ref{lem:1.1.4} (ii) we may assume that $s : B \longrightarrow B^I$ is a cofibration and hence $B^I$ is in $\cat C_c$. 
\end{itemize} 
\end{proof} 

By Lemma~\ref{lem:1.1.8} the functors $\gamma_c$, $\gamma_f$, $\gamma$ induce functors $\xoverline \gamma_c : \pi \cat C_c \longrightarrow \Ho \cat C_c$, $\xoverline \gamma_f : \pi \cat C_f \longrightarrow \Ho \cat C_f$ and $\xoverline \gamma : \pi \cat C_{cf} \longrightarrow \Ho \cat C$, provided these localizations exist. The following result shows that the homotopy category $\Ho \cat C$ as defined in Definition~\plscite{6} is equivalent to the more concrete category $\pi \cat C_{cf}$. 

\begin{customthm}{1'}\label{thm:1'}
$\Ho \cat C$ exists and the functor $\xoverline \gamma : \pi \cat C_{cf} \longrightarrow \Ho \cat C$ is an equivalence of categories. 
\end{customthm} 

This is included in the following more complex assertion which is presented for the purpose of comparison with (Gabriel--Zisman \cite{gabriel_calculus_1967}).

\begin{theorem}\label{thm:1}
The categories $\Ho \cat C$, $\Ho \cat C_c$, $\Ho \cat C_f$ exist and there is a diagram of functors
\[\tag{11}\label{diag:1.1.11}
    \begin{tikzcd}
    \pi \cat C_c \arrow[rr, "\xoverline \gamma_c"]                        &  & \Ho \cat C_c \arrow[dd] \\
                                                                             &  &                             \\
    \pi \cat C_{cf} \arrow[uu] \arrow[dd] \arrow[rr, "\xoverline \gamma"] &  & \Ho \cat C              \\
                                                                             &  &                             \\
    \pi \cat C_f \arrow[rr, "\xoverline \gamma_f"]                        &  & \Ho \cat C_f \arrow[uu]
    \end{tikzcd}
\]
where $\longhookrightarrow{}$ denotes a full embedding and $\varrightarrow{\sim}$ denotes an equivalence of categories. Furthermore, if $(\xoverline \gamma)^{-1}$ is a quasi--inverse for $\xoverline \gamma$, then the fully faithful functor 
\[\Ho \cat C_c \varrightarrow{\sim} \Ho \cat C \varrightarrow[\sim]{(\xoverline \gamma)^{-1}} \pi \cat C_{cf} \longhookrightarrow{} \pi \cat C_c\] is right adjoint to $\xoverline \gamma_c$ and the fully faithful functor \[\Ho \cat C_f \varrightarrow{\sim} \Ho \cat C \varrightarrow[\sim]{(\xoverline \gamma)^{-1}} \pi \cat C_{cf} \longhookrightarrow{} \pi \cat C_f\] is left adjoint to $\xoverline \gamma_f$. 
\end{theorem} 

\begin{proof}
For each object $X$ choose a trivial fibration $p_X : Q(X) \longrightarrow X$ with $Q(X)$ cofibrant and a trivial fibration $i_X : X \longrightarrow R(X)$ with $R(X)$ fibrant. We assume that $Q(X) = X$ and $p_X = \id_X$ (resp. $X = R(X)$ and $i_X = \id_X$) if $X$ is already cofibrant (resp. fibrant). For each map $f : X \longrightarrow Y$ we may choose by \ref{1.1.M1} a map $Q(f) : Q(X) \longrightarrow Q(Y)$ (resp. $R(f) i_X = i_Y f$) which is unique up to left (resp. right) homotopy by Lemma~\ref{lem:1.1.7}. It follows that $Q(gf) \LHom Q(g) Q(f)$ and \\ $Q(\id_X) \LHom \id_{Q(X)}$, hence $Q(gf) \RHom Q(g) Q(f)$ and $Q(\id_X) \RHom \id_{Q(X)}$ by Lemma~\ref{lem:1.1.4}(i) and therefore $X \longrightarrow Q(X)$, $f \mapsto \xoverline {Q(f)}$ is a well--defined functor which we shall denote $\xoverline Q : \cat C \longrightarrow \pi \cat C_c$. Similarly there is a functor $\xoverline R : \cat C \longrightarrow \pi \cat C_f$. 

If $X$ is cofibrant, $f, g \in \Hom(X, Y)$, and $f \RHom g$, then by Lemma~\ref{lem:1.1.4}(iii) $i_Y f \RHom i_Y g$ and hence $R(f) \sim R(g)$ by the dual of Lemma~\ref{lem:1.1.7}. It follows that $\xoverline R$ restricted to $\cat C_c$ induces a functor $\pi \cat C_c \longrightarrow \pi \cat C_{cf}$ and that there is a well--defined functor $\xoverline {RQ} : \xoverline C \longrightarrow \pi \cat C_{cf}$ given by $X \longrightarrow RQX$, $f \mapsto \xoverline {RQ(f)}$.

Let $\Ho \cat C$ be the category having the same objects as $\cat C$ with \[\Hom_{\Ho \cat C}(X, Y) = \Hom_{\pi \cat C_{cf}}(RQX, RQY) = \pi(RQX, RQY)\]
and the obvious composition. Let $\gamma : \cat C \longrightarrow Ho \cat C$ be given by $\gamma(X) = X$, $\gamma(f) = \xoverline {RQ(f)}$. As $RQ(X) = X$ if $X$ is in $\cat C_{cf}$, it is clear that the functor $\xoverline \gamma : \pi \cat C_{cf} \longrightarrow \Ho \cat C$ induced by $\gamma$ is fully faithful. By Lemma~\ref{lem:1.1.7} and its dual, trivial fibrations and trivial cofibration in $\cat C_{cf}$ become isomorphisms in $\pi \cat C_{cf}$; hence any weak equivalence in $\cat C_{cf}$ becomes an isomorphism in $\pi \cat C_{cf}$ by \ref{1.1.M2} and \ref{1.1.M5}. If $f : X \longrightarrow Y$ is a weak equivalence in $\cat C$, then $f p_X = p_Y Q(f)$ and \ref{1.1.M5} imply that $Q(f)$ is a weak equivalence in $\cat C_c$ and similarly $RQ(f)$ is a weak equivalence in $\cat C_{cf}$; hence $\gamma(f) = \xoverline {RQ(f)}$ is an isomorphism. It follows that for any $X$ the maps \[X \xleftarrow{p_X} Q(X) \varrightarrow{i_{Q(X)}} RQX\] yield an isomorphism of $X$ and $RQ(X)$ in $\Ho \cat C$ and hence $\pi \cat C_{cf} \varrightarrow{\xoverline \gamma} \Ho \cat C$ is an equivalence of categories. 

We now show that $\gamma : \cat C \longrightarrow \Ho \cat C$ has the required universal property of Definition~\ref{def:1.1.5}. As mentioned above $\gamma$ carries weak equivalences in $\cat C$ into isomorphisms in $\Ho \cat C$. Let $F : \cat C \longrightarrow \cat B$ do the same. Define $\theta : \Ho \cat C \longrightarrow \cat C$ by $\theta(X) = F(X)$ and for $\alpha \in \Hom_{\Ho \cat C}(X, Y)$ choose $f : RQ(X) \longrightarrow RQ(Y)$ representing $\alpha$ and let $\theta(\alpha)$ be given by the diagram
% https://q.uiver.app/?q=WzAsNixbMCwwLCJGKFgpIl0sWzIsMCwiRihZKSJdLFswLDIsIkYoUVgpIl0sWzIsMiwiRihRWSkiXSxbMCw0LCJGKFJRWCkiXSxbMiw0LCJGKFJRWSkiXSxbMCwxLCJcXHRoZXRhKFxcYWxwaGEpIiwwLHsic3R5bGUiOnsiYm9keSI6eyJuYW1lIjoiZGFzaGVkIn19fV0sWzQsNSwiRihmKSJdLFsyLDAsIkYocF9YKSJdLFsyLDQsIkYoaV97UVh9KSIsMl0sWzMsMSwiRihwX1kpIiwyXSxbMyw1LCJGKGlfe1FZfSkiXSxbMiw0LCJcXHNpbSJdLFsyLDAsIlxcc2ltIiwyXSxbMyw1LCJcXHNpbSIsMl0sWzMsMSwiXFxzaW0iXV0=
\[\tag{12}\label{diag:1.1.12}\begin{tikzcd}
	{F(X)} && {F(Y)} \\
	\\
	{F(QX)} && {F(QY)} \\
	\\
	{F(RQX)} && {F(RQY)}
	\arrow["{\theta(\alpha)}", dashed, from=1-1, to=1-3]
	\arrow["{F(f)}", from=5-1, to=5-3]
	\arrow["{F(p_X)}", from=3-1, to=1-1]
	\arrow["{F(i_{QX})}"', from=3-1, to=5-1]
	\arrow["{F(p_Y)}"', from=3-3, to=1-3]
	\arrow["{F(i_{QY})}", from=3-3, to=5-3]
	\arrow["\sim", from=3-1, to=5-1]
	\arrow["\sim"', from=3-1, to=1-1]
	\arrow["\sim"', from=3-3, to=5-3]
	\arrow["\sim", from=3-3, to=1-3]
\end{tikzcd}\]
By Lemma~\ref{lem:1.1.8}(i), $\theta(\alpha)$ is independent of the choice of $f$ and it is then clear that $\theta$ is a functor, in fact the unique functor with $\theta \circ \gamma = F$. This proves the existence of $\Ho \cat C$ and also the horizontal equivalence in (\ref{diag:1.1.11}).

The existence of $\Ho \cat C_c$ and the equivalence $\pi \cat C_{cf} \varrightarrow{\sim} \Ho \cat C_c$ can be proved in the same way using the functor $\cat C_c \longrightarrow \pi \cat C_{cf}$ induced by $\xoverline R$ and Lemma~\ref{lem:1.1.8}(ii). The last assertion of the theorem results from the fact that the inclusion functor $\pi \cat C_{cf} \longhookrightarrow \pi \cat C_c$ is right adjoint to the functor $\xoverline R' : \pi \cat C_c \longrightarrow \pi \cat C_{cf}$, since \\$\pi^r(X, Y) \cong \pi(RX, Y)$ if $X$ is in $\cat C_c$ and $Y$ is in $\cat C_{cf}$ by Lemma~\ref{lem:1.1.7}, and from the fact that up to the equivalence $\Ho \cat C_c \cong \Ho \cat C \cong \pi \cat C_{cf}$, $\,\xoverline \gamma_c : \cat C_c \longrightarrow \Ho \cat C_c$ ``is'' the functor $\xoverline R'$. 
\end{proof} 

\begin{corollary}
If $A$ is cofibrant and $B$ is fibrant, then \[\Hom_{\Ho \cat C}(A, B) = \pi(A, B)\]
\end{corollary}

\begin{proof}
\[\Hom_{\Ho \cat C}(A, B) = \pi(RQA, RQB) = \pi(RA, QB) \cong \pi(A, QB) \cong \pi(A, B)\] by Lemma~\ref{lem:1.1.7} and its dual. 
\end{proof}

\begin{corollary}
The functor $\xoverline \gamma_c : \pi \cat C_c \longrightarrow \Ho \cat C_c$ permits calculations by left fractions and the functor $\xoverline \gamma_f : \pi \cat C_cf \longrightarrow \Ho \cat C_f$ permits calculation by right fractions. 
\end{corollary} 

\begin{proof}
This follows from the first chapter of \cite{gabriel_calculus_1967}, since $\xoverline \gamma_c$ has a fully faithful right adjoint. 
\end{proof} 

\begin{remarks*}
\begin{enumerate}
    \item In general the localization $\cat C \longrightarrow \Ho \cat C$ cannot be calculated by either left or right fractions.
    \item In example \ref{ex:1.1.A}, $\cat C = \cat C_f$ and the usual homotopy relation on maps coincides with homotopy in the sense of Definition~\ref{def:1.1.5} on $\cat C_c$. Thus $\pi \cat C_{cf} = \pi \cat C_c$ is the homotopy category of cofibrant spaces which in turn is equivalent to the usual homotopy category of CW complexes. In example \ref{ex:1.1.B}, $\cat C = \cat C_f$ and homotopy on $\cat C_c$ coincides with the chain homotopy relation. Hence $\pi \cat C_c = \pi \cat C_{cf}$ is what is denoted by $K^-(\mathbf P)$ is Harshorne \cite{hartshorne_residues_1966} where $\mathbf P$ is the additive sub--category of projectives in $\cat A$, while $\Ho \cat C$ is the derived category $D^-(\cat A)$ or $D_+(\cat A)$. 
    \item The following example shows that although $\Ho \cat C$ is determined by the category $\cat C$ and the class of weak equivalences, the model structure on $\cat C$ isn't. Let $\cat A$ be an abelian category of \defemph{finite} homological dimension having enough projectives and injectives. Then $\cat C = c_b(\cat A)$ the category of bounded complexes is what one should call a full sub--model category of $c_+(\cat A)$ as in example \ref{ex:1.1.B}. The dual of example \ref{ex:1.1.B} gives the structure of a model category on $c_-(\cat A)$, the category of complexes bounded above, where cofibrations are injections, fibrations are surjective maps with injective kernels, and weak equivalences are homology isomorphisms. Again $c_b(\cat A)$ is a full--sub--model category of $c_-(\cat A)$ and we obtain different model structures on $c_b(\cat A)$ with the same family of weak equivalences.
\end{enumerate}
\end{remarks*}
\end{document}