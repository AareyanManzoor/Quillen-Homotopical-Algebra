\documentclass[../main]{subfiles}

\begin{document}
\chapter*{Preface}
\addcontentsline{toc}{chapter}{Preface}

Homotopical algebra or non-linear homological algebra is the
generalization of homological algebra to arbitrary categories
which results by considering a simplicial object as being a generalization of a chain complex. The first step in the theory was presented in \cite{dold_homology_1958}, \cite{dold_homologie_1961}, where the derived functors of a non-additive functor from an abelian category $\mathbf{A}$ with enough projectives to another category $\mathbf{B}$ were constructed. This construction generalizes to the case where $\mathbf{A}$ is a category closed under finite limits having sufficiently many projective objects, and these derived functors can be used to give a uniform definition of cohomology for universal algebras. In order to compute this cohomology for commutative rings, the author was led to consider the
simplicial objects over $\mathbf{A}$ as forming the objects of a homotopy
theory analogous to the homotopy theory of algebraic topology,
then using the analogy as a source of intuition for simplicial
objects. This was suggested by the theorem of Kan \cite{kan_homotopy_1958} that the
homotopy theory of simplicial groups is equivalent to the homotopy theory of connected pointed spaces and by the derived category (\cite{hartshorne_residues_1966}, \cite{verdier_categories_nodate}) of an abelian category. The analogy turned out
to be very fruitful; but there were a large number of arguments which were formally similar to well-known ones in algebraic topology, so it was decided to define the notion of a homotopy theory
in sufficient generality to cover in a uniform way the different
homotopy theories encountered. This is what is done in the present paper; applications are reserved for the future. 

The following is a brief outline of the contents of this paper; for a more complete discussion see chapter introductions. Chapter \ref{ch:1} contains an axiomatic development of homotopy theory patterned on the derived category of an abelian category. In Chapter \ref{ch:2} we
give various examples of homotopy theories that arise from these
axioms, in particular we show that the category of simplicial objects in a category $\mathbf{A}$ satisfying suitable conditions gives rise
to a homotopy theory. Also in \S\ref{sec:2.5} we give a uniform description
of homology and cohomology in a homotopy theory as the ``linearization'' or ``abelianization'' of the non-linear homotopy situation,
and we indicate how in the case of algebras this yields a reasonable cohomology theory.

The author extends his thanks to S. Lichtenbaum and
M. Schlesinger who suggested the original problem on commutative
ring cohomology, to Robin Hartshorne whose seminar \cite{hartshorne_residues_1966} on
Grothendieck's duality theory introduced the author to the derived
category, and to Daniel Kan for many conversations during which
the author learned about simplicial methods and formulated many
of the ideas in this paper.

\end{document}