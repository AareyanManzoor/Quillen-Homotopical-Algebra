\newcommand{\defemph}[1]{\textbf{#1}} %to emphases words underlined in the original text.
\newcommand{\defemphi}[1]{\textbf{#1}\index{#1}}

\DeclareMathOperator{\Ho}{Ho} %homotopy category of e.g. C
\DeclareMathOperator{\Ob}{Ob} %objects of a category 
\DeclareMathOperator{\Hom}{Hom} %morphisms between e.g. two objects A,B \in category C.
\DeclareMathOperator{\Map}{Map} %maps between objects
\DeclareMathOperator{\Ker}{Ker}
\DeclareMathOperator{\Coker}{Coker}
\DeclareMathOperator{\Image}{Im} %image
\DeclareMathOperator{\pr}{pr} %for projection functions.
\DeclareMathOperator{\In}{in}
\DeclareMathOperator{\id}{id} %for identity function
\DeclareMathOperator{\Aut}{Aut} %automorphisms of objects
\DeclareMathOperator{\Ex}{Ex} %whenever there is a Ex or Ex^infty functor in the book.
\DeclareMathOperator{\cosk}{cosk} %coskeleton, see page 126
\DeclareMathOperator{\Ch}{Ch} %chain complexes
\DeclareMathOperator{\Der}{Der}
\newcommand{\homology}{H} %for (co)homology
\newcommand{\chomology}{\check{\homology}}
\newcommand{\uhomology}{\mathbf{H}}
\newcommand{\cuhomology}{\mathbf{\check{H}}}
\DeclareMathOperator{\Tor}{Tor} %Tor
\DeclareMathOperator{\Ext}{Ext}
\DeclareMathOperator{\Sing}{Sing}
\DeclareMathOperator{\ev}{ev}
\DeclareMathOperator{\ab}{ab}
\renewcommand{\epsilon}{\varepsilon}

\newcommand{\cat}[1]{\mathscr{#1}} %for the fancy A,B,C etc used to denote model categories.

\newcommand{\RHom}{\overset{r}{\sim}} %for the equivalence relations of right homotopy
\newcommand{\LHom}{\overset{l}{\sim}} %for the equivalence relations of left homotopy
\newcommand{\UHom}{\operatorname{\mathbf{Hom}}} %underlined Hom

\newcommand{\xoverline}[1]{\mskip.5\thinmuskip\overline{\mskip-.5\thinmuskip {#1} \mskip-.5\thinmuskip}\mskip.5\thinmuskip}
\newcommand{\xhat}[1]{\mskip.5\thinmuskip\widehat{\mskip-.5\thinmuskip {#1} \mskip-.5\thinmuskip}\mskip.5\thinmuskip}
\newcommand{\xtilde}[1]{\mskip.5\thinmuskip\widetilde{\mskip-.5\thinmuskip {#1} \mskip-.5\thinmuskip}\mskip.5\thinmuskip}
\newcommand{\xunderline}[1]{\mskip.5\thinmuskip\underline{\mskip-.5\thinmuskip {#1} \mskip-.5\thinmuskip}\mskip.5\thinmuskip}

%%use this to overset/underset things to arrows, and in general use \varrightarrow{} for maps 

\makeatletter

\def\@@varrightarrow#1#2#3{\begingroup%
\setbox0=\hbox{$#1\xrightarrow[#3]{#2}$}%
\setbox1=\hbox{$#1\longrightarrow$}%
\ifdim\wd0<\wd1 \mathrel{\mathop{\longrightarrow}\limits^{#2}_{#3}}
\else \xrightarrow[#3]{#2} \fi\endgroup}

\def\@varrightarrow#1#2{\@@varrightarrow#1#2}

% \varrightarrow[under material]{over material} puts material over and under a right arrow.
% It automatically prints the longer option between an xrightarrow and a longrightarrow
\renewcommand\varrightarrow[2][]{\mathpalette\@varrightarrow{{#2}{#1}}}


\newcommand{\rightrightrightarrows}{\mathbin{\substack{\rightarrow \\[-1em] \rightarrow \\[-1em] \rightarrow}}} % like \rightrightarrows but with 3 arrows

\newcommand{\dottopright}[1]{\dot{#1}}

\newcommand{\indexline}{---\ } %for index, split entry

\tikzstyle{bullet}=[fill=black, draw=black, shape=circle, tikzit shape=circle, xscale=0.01cm, tikzit draw=black, tikzit fill=black, yscale=0.01cm]

%%%shortcut
\newcommand{\C}{\mathbf{C}}
\newcommand{\bL}{\mathbf{L}}
\newcommand{\bR}{\mathbf{R}}